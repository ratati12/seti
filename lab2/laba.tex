\documentclass[11pt]{article}
\usepackage[T1,T2A]{fontenc}
\usepackage[utf8]{inputenc}
\usepackage[english,russian]{babel}
\usepackage{graphicx}
\usepackage{amsmath}
\graphicspath {{img/}}
\title{\textbf{Лабораторная работа №2\\<<Исследование процессов в проводных линиях связи (ЛС)>>}}
\author{Матяш А.А., ККСО-01-19}
\date{}
\addtolength{\topmargin}{-3cm}
\addtolength{\textheight}{3cm}
\begin{document}
\maketitle
\thispagestyle{empty}
\textbf{Цель работы:} экспериментальное подтверждение волновых процессов в проводных линиях связи, используемых в качестве физической среды при организации каналов передачи данных и приобретение практических навыков постановки и проведения исследований 
\section{Перечень элементов на схемах}
\subsection{<<ЛС в режиме согласованной линии>>}
\begin{itemize}
    \item[-] Четырех канальный осциллограф
    \item[-] Источник переменного тока (5 В, 500 кГц)
    \item[-] Резистор (3.3 кОм)
    \item[-] Двух проводная ЛС с потерями (50 м, 10 Ом)
\end{itemize}
\subsection{<<ЛС с потерями в режиме несогласованной разомкнутой линии>>}
\begin{itemize}
    \item[-] Четырех канальный осциллограф
    \item[-] Источник переменного тока (5 В, 12 МГц)
    \item[-] Двух проводная ЛС с потерями (50 м, 0.001 Ом)
\end{itemize}
\subsection{<<ЛС с потерями в режиме несогласованной замкнутой линии>>}
\begin{itemize}
    \item[-] Источник переменного тока (5 В, 6 Мгц)
    \item[-] Двух проводная ЛС с потерями (50 м, 0.001 Ом)
    \item[-] Двух проводная ЛС с потерями (25 м, 0.001 Ом)
    \item[-] Двух проводная ЛС с потерями (25 м, 0.001 Ом)
    \item[-] Четырех канальный осциллограф
    \item[-] Датчик тока
\end{itemize}
\subsection{<<ЛС с потерями в режиме несогласованной нагрузки>>}
\begin{itemize}
    \item[-] Построитель частотных характеристик
    \item[-] Двух проводная ЛС с потерями (50 м, 1 Ом)
    \item[-] Источник переменного тока (5 В, 500 кГц)
    \item[-] Резистор (0.001 Ом)
    \item[-] Ключ
\end{itemize}
\section{Копии окон схемных файлов с позиционными
обозначениями}
\subsection{<<ЛС в режиме согласованной линии>>}
\includegraphics[width=1\linewidth]{1/scheme.jpg}
\subsection{<<ЛС с потерями в режиме несогласованной разомкнутой линии>>}
\includegraphics[width=1\linewidth]{2/scheme.jpg}
\subsection{<<ЛС с потерями в режиме несогласованной замкнутой линии>>}
\includegraphics[width=1\linewidth]{3/scheme.jpg}
\subsection{<<ЛС с потерями в режиме несогласованной нагрузки>>}
\includegraphics[width=1\linewidth]{4/scheme.jpg}
\section{Результаты расчетов и измерений приборами}
\subsection{<<ЛС в режиме согласованной линии>>}
Определим значения параметров $Z_0, C, G$:\\
$
Z_0 = \sqrt{\frac{L}{C}} = \sqrt{\frac{11,11\cdot 10^{-6}}{1\cdot 10^{-12}}}\approx 3,3\text{ кОм}\\
L*C = \frac{1}{c^2}=11,11\cdot 10^{-18}\Rightarrow C = \frac{11,11\cdot 10^{-18}}{11,11\cdot 10^{-6}} = 1 \text{ пФ}\\
G = \frac{RC}{L} = \frac{1\cdot 10^{-12}}{11,11\cdot 10^{-6}} = 9\cdot 10^{-8} = 90\cdot 10^{-9} = 90 \text{ нСм/м}\\
$
1. Исследуем модель линии связи для различных частот входного сигнала:
\begin{itemize}
    \item Осциллограмма при 500 кГц\\
        \includegraphics[width=1\linewidth]{1/500kgzosc.jpg}
    \item Осциллограмма при 800 кГц\\
        \includegraphics[width=1\linewidth]{1/800kgzosc.jpg}
\end{itemize}
2. Осциллограмма при R = 10 Ом/м:\\
\includegraphics[width=1\linewidth]{1/10ommosc.jpg}
3. Запаздывание выходного сигнала относильно входного ($T_2-T_1$):
\begin{itemize}
    \item Для 500 кГц: \\$\tau_1 = 147*10^{-9}$с
    \item Для 800 кГц: \\$\tau_2 = 163*10^{-9}$с
\end{itemize}
4. Определеим запаздывание выходного сигнала относительно входного на длину линии в режиме бегущей волны:\\
$\beta = 2*\pi*f(T_2-T_1) = 2*\pi*\tau \Rightarrow \\ 
\Rightarrow \beta_1 = 2 * 3.14 * 147 * 10^{-9} = 9,23 * 10^{-6}\\ 
\beta_2 = 2 * 3.14 * 163 * 10^{-9} = 10,24 * 10^{-6}\\ 
$
5. Амплитуды входного $U_1$ и выходного напряжения $U_2$:\\
$U_1 = 7,05$В\\
$U_2 = 6,91$В\\
6. Получим $\alpha$, $\beta l$ и $U$:\\
$\beta = \omega * \sqrt{LC} = 500 * 10^3\sqrt{11,11*10^{-6}* 10^{-12}} \approx 166* 10^{-5}\\
\alpha = \sqrt{RG} = \sqrt{10 * 900 * 10^{-9}} = 3 * 10^{-3}\\
U(t) = U_i(t)e^{-\alpha l}cos\omega t - \beta l = 7 * e^{-50 * 3 * 10^{-3}} * cos(-166 * 10^{-5} * 50) = 6$В
\subsection{<<ЛС с потерями в режиме несогласованной разомкнутой линии>>}
\includegraphics[width=1\linewidth]{2/2osc.jpg}
Запаздывание выходного сигнала относительно входного:\\
$T_2 - T_1 = 162$нС\\
Амплитуды входного и выходного напряжений:\\
$U_i = 6$В\\
$U = 14$В
\subsection{<<ЛС с потерями в режиме несогласованной замкнутой линии>>}
\includegraphics[width=1\linewidth]{3/3osc.jpg}
Запаздывание выходного сигнала относительно входного:\\
$T_2 - T_1 = 83$нС\\
Амплитуды входного и выходного напряжений:\\
$U_i = 7$В\\
$U = 7$В
Выходной ток:\\
$I = 800$мА\\
\subsection{<<ЛС с потерями в режиме несогласованной нагрузки>>}
\begin{itemize}
    \item[-] Ключ разомкнут:\\
        \includegraphics[width=1\linewidth]{4/4open.jpg}
    \item[-] Ключ замкнут:\\
        \includegraphics[width=1\linewidth]{4/4close.jpg}
\end{itemize}
\textbf{Вывод:} в этой работе мы ознакомились с теорией волновых процессов в проводных линиях связи, исследовали режимы бегущих и стоячих волн.
\end{document}

